%%%%%%%%%%%%%%%%%%%%%%%%%%%%%%%%%%%%%%%%%
% Medium Length Graduate Curriculum Vitae
% LaTeX Template
% Version 1.1 (9/12/12)
%
% This template has been downloaded from:
% http://www.LaTeXTemplates.com
%
% Original author:
% Rensselaer Polytechnic Institute (http://www.rpi.edu/dept/arc/training/latex/resumes/)
%
% Important note:
% This template requires the res.cls file to be in the same directory as the
% .tex file. The res.cls file provides the resume style used for structuring the
% document.
%
%%%%%%%%%%%%%%%%%%%%%%%%%%%%%%%%%%%%%%%%%
%----------------------------------------------------------------------------------------
%       PACKAGES AND OTHER DOCUMENT CONFIGURATIONS
%----------------------------------------------------------------------------------------
\documentclass[margin, 10pt]{res} % Use the res.cls style, the font size can be changed to 11pt or 12pt here
\usepackage{helvet} % Default font is the helvetica postscript font
\usepackage{hyperref} 
%\usepackage{newcent} % To change the default font to the new century schoolbook postscript font uncomment this line and comment the one above
\setlength{\textwidth}{5.3in} % Text width of the document
\begin{document}
%----------------------------------------------------------------------------------------
%       NAME AND ADDRESS SECTION
%----------------------------------------------------------------------------------------
\moveleft.5\hoffset\centerline{\large\bf Brinden Carlson} % Your name at the top
 
\moveleft\hoffset\vbox{\hrule width\resumewidth height 1pt}\smallskip % Horizontal line after name; adjust line thickness by changing the '1pt'
 
%\moveleft.5\hoffset\centerline{3080 } % Your address
%\moveleft.5\hoffset\centerline{Chicago, Illinois 60616}
\moveleft.5\hoffset\centerline{(970)-433-3380 \textbullet \  bcarlson1@ufl.edu \textbullet \ \href{https://bear-is-asleep.github.io/}{https://bear-is-asleep.github.io/}}
%\textbullet \href{https://github.com/bear-is-asleep}{GitHub: bear-is-asleep}}
%----------------------------------------------------------------------------------------
\begin{resume}
%----------------------------------------------------------------------------------------
%       EDUCATION SECTION
%----------------------------------------------------------------------------------------
\section{EDUCATION}
{\sl PhD Candidate,} Physics \\
University of Florida, Gainesville, FL, expected 2025 \\
GPA: 3.74/4.0 \\

{\sl Bachelor of Science,} Physics and Astrophysics \\
Illinois Institute of Technology, Chicago, IL, May 2020 \\
GPA: 3.69/4.0 (7 $\times$ Dean's List)
%Physics GPA: 3.74/4.0 \\ 

%----------------------------------------------------------------------------------------
%       HONORS AWARDS
%----------------------------------------------------------------------------------------
\section{HONORS/ \\ AWARDS} 

%\href{https://www.phys.ufl.edu/wp/index.php/2024/10/01/physics-graduate-student-is-selected-for-department-of-energy-office-of-science-research-program/}{
DOE SCGSR award at SLAC \hfill Jan 2025 - July 2025 \\
Tom Scott Memorial Award \hfill Dec 2024 \\
URA VSP award \hfill Oct 2024 - Dec 2024 \\
Best Talk of session at New Perspectives 2024 \hfill July 8, 2024 \\
Member of Sigma Xi Research Honor Society \hfill Fall 2024 - Present \\
UF Honors Society \hfill Spring 2021 - Present\\
UF IHEPA fellowship \hfill Summer 2021\\
Member of $\Sigma \Pi \Sigma$ Physics Honors Society \hfill Spring 2019 - Spring 2020\\
Secretary of ACS at Illinois Institute of Technology \hfill Fall 2019 - Spring 2020\\
%2 years collegiate varsity basketball (1$\times$ national championship appearance) \\
Member of SPS at Illinois Institute of Technology \hfill Fall 2018 - Spring 2020 \\

%----------------------------------------------------------------------------------------
%       COMPUTER SKILLS SECTION
%----------------------------------------------------------------------------------------
%\section{SKILLS} 
%{\sl Languages \& Software:} 
%Python (NumPy, matplotlib, TensorFlow, Keras, Pandas, Pytorch), GitHub, C++, Root, OASYS (Shadow, XOP), Mathematica, Octave, Microsoft Office, Libre Office, G4beamline, Matlab, Swift, LaTex, AutoCAD (Inventor), Historoot, Raspberry Pi3, Arduino, Java. \\
%{\sl Operating Systems:} Unix, Windows. \\
%{\sl Frameworks:} GENIE, LArSoft, 
 
%----------------------------------------------------------------------------------------
%       GRADUATE RESEARCH
%----------------------------------------------------------------------------------------
 
\section{GRADUATE \\ RESEARCH}

{\sl SBND, Fermi National Laboratory} \hfill Fall 2020 - Present \\
\hfill University of Florida \\
\hfill Experimental Neutrino Physics, Dr. Heather Ray Advising
\begin{itemize}
    \item Created a GENIE tune with HF-CRPA nuclear modeling to study SBND-PRISM and compare to existing nuclear model in neutrino-nucleus interactions.
    \item Utilized SBND-PRISM to study phenomenological models to replicate the T2K ND280 and a step-function flux spectrum.
    \item Implemented flash matching into machine learning-based reconstruction of neutrino events to identify neutrino events with 90\% purity and 92\% efficiency.
    \item  Compared final state muon kinematics from various nuclear models within GENIE with existing MicroBooNE data using NUISANCE.
    \item Utilized neutrino beam angular dependence (PRISM) to identify observable differences between different nuclear models in SBND.
    \item Designed reconstruction selection cuts to isolate a sample of Muon Neutrino Charged Current events.
    \item Ported machine learning code, generated training sample, optimized hyperparameters, and trained machine learning reconstruction algorithm of neutrino events in LArTPCs and achieved above 90\% primary particle ID efficiency.
    \item Created PDS event display to verify mapping of each PDS component into its expected location in the hardware database by comparing the cumulative photons of each PD to its neighbors and the overlayed reconstructed track locations.
    \item Performed analysis of using neutrino electron scattering events to constrain neutrino flux at SBND.
    \item Modified GENIE neutrino event generator to account for orbital electron's motion on cross section and final state electron's kinematics.
    \item Installed PMTs, X-ARAPUCAs, and calibration fibers onto SBND frame and performed quality control checks throughout installation phase to complete detector installation of the PDS.
    \item Measured PMT waveforms in test stands and fixed malfunctioning PMTs by replacing fried resistors, missing capacitors, and resoldering cable to PMT base to ensure proper functionality.
\end{itemize}

%----------------------------------------------------------------------------------------
%       UNDERGRADUATE RESEARCH
%----------------------------------------------------------------------------------------
 
\section{UNDER-\\GRADUATE \\ RESEARCH}

{\sl NOvA Test Beam, Fermi National Laboratory} \hfill Fall 2017 - Spring 2020 \\
\hfill Illinois Institute of Technology \\
\hfill Research Advisor: Dr. Daniel Kaplan \& Dr. Yagmur Torun
\begin{itemize}
    \item Employed a particle simulation using G4beamline to study how a low-energy proton beam creates a beam of tagged neutrons with known energy and position in order to calibrate a detector.
    %\item Calculated outgoing neutron kinetic energy using conservation of momentum to evaluate expected resolution of tagged neutron beam.
    \item  Conducted tests on various sets of magnetic field data from the dipole spectrometer magnet employed in the NOvA Test Beam to benchmark its performance.
    \item Ran tests on time of flight (TOF) cable delay times with a group of graduate assistants by recording statistics on waveforms produced from a Pulser machine to calibrate timing delays in experiment.
\end{itemize}
{\sl Advanced Photon Source, Argonne National Laboratory} \hfill Fall 2018 - Fall 2019 \\
\hfill Illinois Institute of Technology \\
\hfill Research Advisor: Dr. Ali Khounsary
\begin{itemize}
    \item  Simulated sundry optical configurations for a x-ray beam created via synchrotron radiation from a dipole bending magnet.
    \item Analyzed intensity and energy resolution for a sagittally bent monochromator's by varying the sagittal radius and selected beam energy. 
\end{itemize}


%----------------------------------------------------------------------------------------
%  POSITIONS AND TEACHING
%---------------------------------------------------------------------------------------- 
\section{POSITIONS/\\TEACHING}
{\sl SBND Machine Learning Reconstruction Group Convener, Fermi National Laboratory}
\hfill Fall 2024 - Present
Fermi National Laboratory, Batavia, IL
\begin{itemize}
    \item Communicate group results to SBND collaboration.
    \item Assist group members with SBND machine learning projects by providing technical support.
    \item Lead group meetings by reviewing group member results and providing feedback.
\end{itemize}

{\sl SBND Software Release Manager, Fermi National Laboratory}
\hfill Summer 2024 - Present
Fermi National Laboratory, Batavia, IL
\begin{itemize}
    \item Developed standardized contributing guidelines to ensure robust code integration and experiment-wide communication.
    \item Triggered continuous integration (CI) tests for each pull request and created tagged releases by triggering in Jenkins.
    \item Deployed tagged releases on a common build node to distribute software access to all collaborators.
\end{itemize}

{\sl Research Assistant, UF Neutrino Group}
\hfill Summer 2023 - Present \\
Fermi National Laboratory, Batavia, IL
\begin{itemize}
    \item Assisted undergraduate students in studies at SBND such as single kaon production, PMT cable delays, and PMT Neutral Current software triggering metrics.
\end{itemize}

% {\sl Teaching Assistant, University of Florida}
% \hfill Fall 2020 - Fall 2021\\
% University of Florida, Gainesville, FL
% \begin{itemize}
%     \item Created and executed itinerary based on course schedule for four discussion sections per week.
%     \item Designed questions for weekly discussion session quizzes.
%     \item Ran classroom lab sections across three classrooms simultaneously to ensure social distancing and to assist students in understanding and executing labs (Spring 2021 only).
% \end{itemize}

% {\sl Teaching Assistant, Illinois Institute of Technology}
% \hfill Fall 2019 - Spring 2020\\
% Illinois Institute of Technology, Chicago, IL
% \begin{itemize}
%     \item  Assisted students in designing and learning about circuits and their components during physics lab sessions.
%     \item Worked with students during class and office hours on computational physics lab exercises.
% \end{itemize}

\section{PRESENT-\\ATIONS}% \\PUBLICATIONS}
\begin{itemize}
    \item \href{https://indico.slac.stanford.edu/event/9890/}{``SBND Machine-Learning Based Data Reconstruction Chain"} SLAC FPD invited seminar, Menlo Park, CA, July 1, 2025
    \item \href{https://inpa.lbl.gov/event/inpa-seminar-speaker-tba-3/}{``Machine-Learning-Based Data Reconstruction Chain for the Short Baseline Near Detector"} LBNL INPA invited seminar, Berkley, CA, June 13, 2025
    \item \href{https://indico.slac.stanford.edu/event/8926/}{SPINE Workshop 2024}, Boston MA, July 22-26, 2024
    \subitem ``LArCV File Making at SBN"
    \subitem ``Particle Identification" 
    \item \href{https://indico.fnal.gov/event/64625/contributions/295270/}{``Studying Neutrino-Nucleus Interactions at SBND"} New Perspectives 2024, Batavia IL, July 8-9, 2024
    \item \href{https://indico.phys.ethz.ch/event/113/contributions/865/}{``SPINE SBND"} NPML 2024, ETH Zurich, June 25-28, 2024
    \item "SBND Machine Learning Reconstruction Chain" \href{https://indico.fnal.gov/event/59757/}{SBN Analysis Workshop}, Batavia IL, July 24-29, 2023
    \item "SBND Implementation" \href{https://indico.slac.stanford.edu/event/7979/}{ICARUS ML Workshop}, Fort Collins CO, July 10-14, 2023 
    \item \href{https://indico.fnal.gov/event/59506/contributions/269960/}{``Neutrino Electron Scattering for Flux Constraint on SBND"} New Perspectives 2023, Batavia, IL, June 26, 2023
    \item \href{https://meetings.aps.org/Meeting/APR23/Session/N10.6}{``Neutrino Electron Scattering for Flux Constraint on SBND"} APS, Minneapolis MN, April 17, 2023
    % \item ``Optimization of X-Ray Focusing using Mirrors at APS" Denver X-Ray Conference, Lombard, IL, August 8, 2019
    % \item \href{https://absuploads.aps.org/presentation.cfm?pid=15219}{``Optimizing Momentum and Time Resolution for the NOvA Test Beam"} APS, Denver CO, April 15, 2019
    % \item \href{https://meetings.aps.org/Meeting/APR19/Session/K01.13}{``NOvA Neutron Test Beam"} APS, Denver CO, April 14, 2019
    
\end{itemize}

%Presented at AmBCP Poster day at Illinois Institute of Technology (2 posters Summer 2018, 3 posters Summer 2019)\\

%\section{LECTURES/\\WORKSHOPS}
% Quantum Mechanics I
% \hfill Classical Mechanics\\
% Electromagnetism I
% \hfill Quantum Mechanics II\\
% Statistical Mechanics
% \hfill Electromagnetism II\\
% Cosmology 
% \hfill Particle Physics I\\
% Machine Learning in Physics
% \hfill Quantum Field Theory I\\
% General Relativity \hfill Computational Physics\\
% Quantum Field Theory II
%~\\ \\\href{https://npc.fnal.gov/neutrino-university}{Neutrino University at Fermilab} \hfill Summer 2018-2023 


%----------------------------------------------------------------------------------------
\end{resume}
\end{document}
